\input{usepackage.tex}

\begin{document}

\section{Theoretische Grundlagen}

Die molekulare Diffusion durch einen Querschnitt wird durch das $1.$ Ficksche Gesetz \ref{1} beschrieben.  Wobei $J$ der Stofftransport, $A$ die Fläche, durch die der Strofftransport erfolgt, $n$ die Stoffmenge, $t$ die Zeit, $c$ die Konzentration, $z$ die Länge des Gradienten und $D$ der Diffusionskoeffizient sind.

\begin {equation}
J=\frac{1}{A}\frac{dn}{dt}=-D\frac{dc}{dz}
\end {equation}

Der Stofftransport wird in einem Stefan-Rohr gemessen, welches So aufgebaut ist, dass an ein temperiertes Rohr mit einem konstanten Gasfluss im rechten Winkel ein Rohr angeschlossen ist in dem sich die zu untersuchende Flüssigkeit befindet. Direkt oberhalb des Flüssigkeitsspiegels ist die Konzentration gleich dem Sättigungsdampfdruck und dort wo die Rohre ineinander gehen gleich Null, da das Isopentan durch den Gasstrom sofort weggespült wird. 

Durch das Verdampfen der Flüssigkeit sinkt der Flüssigkeitsspiegel und steigt die Länge des Gradienten. Da die beiden Partialdrücke jedoch konstant bleiben sinkt somit der Gradient.  Es gilt für den Diffusionskoeffizienten in Abhängigkeit zum Füllstand des Röhrchens Gleichung \ref{2}

\begin {equation}
D_{AB}=-\frac{RT\rho_A}{2p^0M_A}\cdot\frac{z^2-{z_0}^2}{t} \cdot\left[ln\left(1-\frac{p^D}{p^0}\right)\right]^{-1}
\end{equation}

Hierbei ist $R$ die allgemeine Gaskonstante, $T$ die Temperatur, $\rho_A$ die Dichte der zu untersuchenden Substanz, $p^0$ der äußere Druck, $M_A$ die molare Masse, $z$ die Länge des Gradienten zum Zeitpunkt $t$ und $z_0$ die Länge des Gradienten zum Zeitpunkt $t=0$. 

Durch zusammenfassen konstanten Beiträge in Gleichung \ref{2} in einer Konstante $K$ ergibt sich Gleichung \ref{3}.

\begin {equation}
D_{AB}=-K\cdot\frac{z^2-{z_0}^2}{\Delta t} 
\end{equation}

Diese kann nun nach $z$ aufgelöst werden.

\begin {equation}
z=\sqrt{{z_0}^2-\frac{D_{AB}}{K}\cdot\Delta t} 
\end{equation}

Da K durch den Logarithmus in negativ ist steigt z und somit sinkt der Flüssigkeitsstand, was für einen verdampfenden Stoff zu erwarten ist.



\end{document}


