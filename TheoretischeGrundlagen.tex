%\input{usepackage.tex}
%\begin{document}
\setlength\abovedisplayshortskip{20pt}
\setlength\belowdisplayshortskip{20pt}
\setlength\abovedisplayskip{20pt}
\setlength\belowdisplayskip{20pt}
\section{Theoretische Grundlagen}
Die molekulare Diffusion entlang eines Gradienten wird durch das erste Ficksche Gesetz \ref{eq:fickgesetz} ausgedrückt. Hierbei wird das statistische Mittel der gerichteten Teilchenbewegung durch den Fluss der Teilchen beschrieben. Im vorliegenden Fall betrachtet man die klassische Ficksche Diffusion. 
\begin{equation}
J=\frac{1}{A}\frac{dn}{dt}=-D\frac{dc}{dz}
\label{eq:fickgesetz}
\end{equation}
Wobei $J$ der Stofftransport, $A$ die Fläche, durch die der Strofftransport erfolgt, $n$ die Stoffmenge, $t$ die Zeit, $c$ die Konzentration, $z$ die Länge des Gradienten und $D$ der Diffusionskoeffizient sind. Der Stofftransport wird in einem Stefan-Rohr gemessen, welches so aufgebaut ist, dass an ein temperiertes Rohr mit einem konstanten Gasfluss im rechten Winkel ein Rohr angeschlossen ist, in dem sich die zu untersuchende Flüssigkeit befindet. Direkt oberhalb des Flüssigkeitsspiegels ist der Partialdruck des Isopentan gleich dem Sättigungsdampfdruck, dort wo die Rohre ineinander gehen ist der Partialdruck Null, da das gesamte Isopentan durch den Luftstrom sofort abtransportiert wird. Dies erzeugt den Partialdruckgradienten.\\
\\
Durch das Verdampfen der Flüssigkeit sinkt der Flüssigkeitsspiegel und steigt die Länge des Gradienten. Da die beiden Partialdrücke jedoch konstant bleiben sinkt somit der Gradient.  Es gilt für den Diffusionskoeffizienten in Abhängigkeit zum Füllstand des Röhrchens Gleichung \ref{eq:diffkonst}
\begin{equation}
D_{AB}=-\frac{RT\rho_A}{2p^0M_A}\cdot\frac{z^2-{z_0}^2}{t} \cdot\left[ln\left(1-\frac{p^D}{p^0}\right)\right]^{-1}
\label{eq:diffkonst}
\end{equation}
Hierbei ist $R$ die allgemeine Gaskonstante, $T$ die Temperatur, $\rho_A$ die Dichte der zu untersuchenden Substanz, $p^0$ der äußere Druck, $M_A$ die molare Masse, $z$ die Länge des Gradienten zum Zeitpunkt $t$ und $z_0$ die Länge des Gradienten zum Zeitpunkt $t=0$. Durch zusammenfassen der konstanten Beiträge in Gleichung \ref{eq:diffkonst} in $K$ ergibt sich Gleichung \ref{eq:diffkonstmitk}.
\begin{equation}
D_{AB}=-K\cdot\frac{z^2-{z_0}^2}{\Delta t} \quad \quad \text{mit :  } K=\frac{RT\rho_A}{2p^0M_A} \cdot\left[ln\left(1-\frac{p^D}{p^0}\right)\right]^{-1}
\label{eq:diffkonstmitk}
\end{equation}
Diese kann nun nach $z$ aufgelöst werden : 
\begin{equation}
z=\sqrt{{z_0}^2-\frac{D_{AB}}{K}\cdot\Delta t} 
\label{eq:laenge}
\end{equation}
\\
Die Füllhöhe des Flüssigkeitsreservoirs ist folglich nicht linear von der Zeit abhängig, da sich die Abnahme gemäß der Wurzelfunktion mit der Zeit verlangsamt. Diese Verlangsamung gewinnt mit sinkendem Pegel zunehmend an Bedeutung. Da K durch den Logarithmus negativ ist steigt z, was bedeutet, dass der Flüssigkeitsstand sinkt, was für einen verdampfenden Stoff zu erwarten ist.
%\end{document}


