%%\documentclass[a4paper, 12pt]{scrreprt}

\documentclass[a4paper, 12pt]{scrartcl}
%usepackage[german]{babel}
\usepackage{microtype}
%\usepackage{amsmath}
%usepackage{color}
\usepackage[utf8]{inputenc}
\usepackage[T1]{fontenc}
\usepackage{wrapfig}
\usepackage{lipsum}% Dummy-Text
\usepackage{multicol}
\usepackage{alltt}
%%%%%%%%%%%%bis hierhin alle nötigen userpackage
\usepackage{tabularx}
\usepackage[utf8]{inputenc}
\usepackage{amsmath}
\usepackage{amsfonts}
\usepackage{amssymb}

%\usepackage{wrapfig}
\usepackage[ngerman]{babel}
\usepackage[left=25mm,top=25mm,right=25mm,bottom=25mm]{geometry}
%\usepackage{floatrow}
\setlength{\parindent}{0em}
\usepackage[font=footnotesize,labelfont=bf]{caption}
\numberwithin{figure}{section}
\numberwithin{table}{section}
\usepackage{subcaption}
\usepackage{float}
\usepackage{url}
%\usepackage{fancyhdr}
\usepackage{array}
\usepackage{geometry}
%\usepackage[nottoc,numbib]{tocbibind}
\usepackage[pdfpagelabels=true]{hyperref}
\usepackage[font=footnotesize,labelfont=bf]{caption}
\usepackage[T1]{fontenc}
\usepackage {palatino}
%\usepackage[numbers,super]{natbib}
%\usepackage{textcomp}
\usepackage[version=4]{mhchem}
\usepackage{subcaption}
\captionsetup{format=plain}
\usepackage[nomessages]{fp}
\usepackage{siunitx}
\sisetup{exponent-product = \cdot, output-product = \cdot}
\usepackage{hyperref}
\usepackage{longtable}
\newcolumntype{L}[1]{>{\raggedright\arraybackslash}p{#1}} % linksbündig mit Breitenangabe
\newcolumntype{C}[1]{>{\centering\arraybackslash}p{#1}} % zentriert mit Breitenangabe
\newcolumntype{R}[1]{>{\raggedleft\arraybackslash}p{#1}} % rechtsbündig mit Breitenangabe
\usepackage{booktabs}
\renewcommand*{\doublerulesep}{1ex}
\usepackage{graphicx}


\usepackage[backend=bibtex, style=chem-angew, backref=none, backrefstyle=all+]{biblatex}
\bibliography{Literatur.bib}
\defbibheading{head}{\section{Literatur}\label{sec:Lit}} 
\let\cite=\supercite
%\begin{document}
\section{Zusammenfassung}
In der folgenden Tabelle wurden zusammenfassend die bestimmte gemittelte Diffusionskonstante, dessen Fehler, die Standardabweichung sowie der gegebene Literaturwert gegenübergestellt.
\begin{table}[H]
	\centering
	\label{Ergebnisse}
	\caption{Ergebnistabelle der bestimmten Diffusionskonstante als Mittelwert, dessen Fehler sowie Standartabweichung. Insbesondere Vergleich mit Literaturwerten. \cite{Skript} \\}
	\begin{tabular}{C{0.2\linewidth}|C{0.2\linewidth}|C{0.2\linewidth}|C{0.3\linewidth}}
			\hline \addlinespace[1ex] 
		$\overline{D_{AB}} \quad  \left[ \,\si{\frac{m^2}{s}} \right]$  &  $\Delta \overline{D_{AB}} \quad  \left[ \,\si{\frac{m^2}{s}} \right]$ & $\sigma \quad  \left[ \,\si{\frac{m^2}{s}} \right]$ & Literaturwert $ \left[ \,\si{\frac{m^2}{s}} \right]$\\
		\hline \addlinespace[1ex] 
		$8.47\cdot 10^{-6}$ &$1.58\cdot10^{-6}$& $1.05 \cdot 10^{-6}$& $7.92 \cdot 10^{-6}$ \\
			\hline \addlinespace[1ex] 
	\end{tabular}
\end{table}
Der erhaltene experimentelle Wert stimmt in guter Näherung mit dem Literaturwert überein. Insbesondere liegt dieser in dem ersten Standardabweichungsintervall. Der erhaltene Fehler ist auffällig hoch, was vor allem durch die großen Messungenauigkeit der Abschätzung des Füllstandes begründet ist. Im Allgemeinen ist ebenfalls anzumerken, dass die verwendete Dichte von $ 0.62\cdot 10^3\,\left[ \si{\frac{kg}{m^3}}\right] $, gerade für eine Temperatur von $T=20^oC$ gilt. Es ist also zu vermuten, dass sich die Dichte bei der Arbeitstemperatur, welche um $6^oC$ abweicht, verringert. Berechnungen mit einer geringeren Dicht, z.B von nur $0.58\cdot 10^3\,\left[ \si{\frac{kg}{m^3}}\right] $, zeigten eine deutlich bessere Übereinstimmung der erhaltenen gemittelten Diffusionskonstante zum Literaturwert. Ferner kann der vorliegende Fehler durch ein Arbeitsgefäß mit einer feineren Messskala für den Füllstand verringert werden.
%\end{document}


