%\input{usepackage.tex}
%\begin{document}
\section{Durchführung}
Bei einem Aussendruck von $1013.3\,[\si{bar}]$ wurde bei einer Temperatur von $26\,\si{^oC}$ die zeitliche Abnahme des Volumen von Isopentan basierend auf der Gasdiffusion gemessen. Das Stefan-Rohr als experimenteller Aufbau wurde wie folgt verwendet : 


\begin{figure}[H]

\centering
\includegraphics[width=0.8\linewidth]{stefanrohr.png}
\caption{Aus dem Praktikumsskript übernommene Skizze eines Stefan-Rohr. Als Flüssigkeit wurde Isopentan verwendet. Das Volumen sinkt durch Verdampfung kontinuierlich. Dies macht sich darin bemerkbar, das der Abstand z sich ständig vergrößert.}
\end{figure}

Zur Erzeugung des Luftstroms im Strömungsrohr wurde die Auslasseite einer Membranpumpe verwendet. Das Strömungsrohr wurde in eine Silikonöl getaucht und anhand der  Blasengeschwindigkeit von ca zwei Blasen pro Sekunde die Strömungsgeschwindigkeit eingestellt und konstant gehalten. Der Minuskus des vorgelegten Inopentans im höhenverstellbaren Rohres wurde auf $5\,[\si{cm}]$, bezogen auf die geschätzte Mitte des Strömungsrohres, eingestellt.  Das Volumen des Isopentan wurde mittels einer an dem engen Rohr gegebener Skala unter Verwendung einer Lupe alle 5 Minuten abgelesen. Die Messreihe begann ungefähr acht Minuten nach dem Einschalten der Membranpumpe. Anschließend konnten über einen Zeitraum von $85$ Minuten insgesamt 17 Messwerte erhalten werden.
\newpage
%\end{document}
