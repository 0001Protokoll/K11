\input{usepackage.tex}
\begin{document}

\section{Einleitung}

Die Diffusion beschreibt die Verteilung eines oder mehrerer Stoffe in einem Anderen ohne äußere Einwirkungen. Sie beruht auf der Brownschen Molekularbewegung bei der die Teilchen aufgrund von thermischer Energie ungerichtet und zufällig Bewegen. Sofern ein Konzentrationsgradient vorliegt, bewegen sich staistisch mehr Teilchen  vom Ort der höheren Konzentration zum Ort der niedrigeren Konzentration woraus ein Nettostofftransport resultiert. Im Folgenden wird die Gasdiffusion von Isopentan in einem Staefan-Rohr untersucht.


\end{document}
